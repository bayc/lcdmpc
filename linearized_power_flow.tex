\documentclass{article}
\usepackage[utf8]{inputenc}
\usepackage[margin=1.0in]{geometry}
\usepackage{amsmath}
\usepackage{amssymb}
\usepackage{mathtools}
\usepackage{tikz}
\usepackage{circuitikz}
\usepackage{algorithm}
\usepackage{algpseudocode}
\usepackage{hyperref}
\usepackage{tkz-euclide}
\usepackage{nicefrac}
\usepackage{changepage}
\usetikzlibrary{arrows.meta}
%\usepackage{graphicx}
%\usepackage{caption}
%\usepackage{subcaption}
%\graphicspath{{figures/}}
%\usepackage{soul}
%\usepackage{color}
%\usepackage{array,booktabs}
%\newcolumntype{M}[1]{>{\centering\arraybackslash}m{#1}}
%\newcolumntype{P}[1]{>{\raggedright\arraybackslash}p{#1}}

\makeatletter
\def\BState{\State\hskip-\ALG@thistlm}
\makeatother
\algnewcommand{\Initialize}[1]{%
  \State \textbf{initialize:}
  \Statex \hspace*{\algorithmicindent}\parbox[t]{.8\linewidth}{\raggedright #1}
}

\newcommand{\smis}{\ensuremath{\tilde{\mathbf{s}}}}


\newcommand{\tran}{^{\mathsf{T}}}

% I don't like the mathfrak notation
\renewcommand{\Re}{\ensuremath{\operatorname{Re}}} 
\renewcommand{\Im}{\ensuremath{\operatorname{Im}}}

\newcommand{\im}{\ensuremath{^{\mathrm{i}}}}
\newcommand{\re}{\ensuremath{^{\mathrm{r}}}}

\begin{document}
	
I'm using this to record a slightly different technique for linearizing the 
power flow equations for the purpose of LCDMPC.

In contrast to my notation from other documents, I'll let 
$\mathbf{s} \in \mathbb{C}^{N+1}$ \emph{include} the slack node complex power, 
although I'll still say there are $N$ load nodes and one slack node (in the 
first position of $\mathbf{s}$). Similarly, $\mathbf{v} \in \mathbb{C}^{N+1}$
includes the slack node voltage. I'll still use the prime notation for the real
versions of these, so that 
$\mathbf{s}' \coloneqq \begin{bmatrix} \Re(\mathbf{s})\tran & 
\Im(\mathbf{s})\tran \end{bmatrix}\tran \in \mathbb{R}^{2(N+1)}$, as with 
$\mathbf{v}'$.

The complex power flow equation is
\begin{equation}\label{eq:complex_power_flow}
    \mathbf{s} = \mathrm{diag}(\mathbf{v})\bar{\mathbf{Y}}\bar{\mathbf{v}}.
\end{equation}
where $\mathbf{Y}$ is the system admittance matrix and the real version 
of this is
\begin{equation}\label{eq:power_flow_prime}
    \mathbf{s}' = M(\mathbf{v}')\mathbf{Y}'\mathbf{v}'
\end{equation}
where
$$ \mathbf{Y}' \coloneqq \begin{bmatrix}     
    \Re(\mathbf{Y}) &
    -\Im(\mathbf{Y}) \\
    \Im(\mathbf{Y}) &
    \Re(\mathbf{Y})\end{bmatrix} $$
and
$$ M({\mathbf{v}'}) \coloneqq \begin{bmatrix} 
    \mathrm{diag}(\Re(\mathbf{v})) &
    \mathrm{diag}(\Im(\mathbf{v})) \\
    \mathrm{diag}(\Im(\mathbf{v})) &
    -\mathrm{diag}(\Re(\mathbf{v}))\end{bmatrix}. $$
Here, I've used the convention that $\mathrm{diag}(x), x\in\mathbb{R}^n$ is an 
$n\times n$ real matrix with the elements of $x$ on the lead diagonal and
zeros elsewhere.

Equation \eqref{eq:power_flow_prime} is nonlinear and so can't be used in a 
convex MPC formulation. As an approximation, we linearize according to 
\begin{align}\label{eq:Linearized_power_flow}
    \mathbf{s}' &\approx \mathbf{s}'_* + 
                        \left.\frac{\partial \mathbf{s}'}
                        {\partial \mathbf{v}'}\right\rvert_
                        {\mathbf{v} = \mathbf{v}'}
                        \left(\mathbf{v}' - \mathbf{v}_*'\right) \nonumber \\
                &= \mathbf{s}'_* + \mathbf{J}
                   \left(\mathbf{v}' - \mathbf{v}_*'\right)\:.
\end{align}
where $(\mathbf{s}_*', \mathbf{v}_*')$ are a linearization point that satisfies
the power flow equation \eqref{eq:power_flow_prime}.
The derivation of the Jacobian $\mathbf{J}$ is listed in the Appendix.

For the LCDMPC problem, we need Equation~\eqref{eq:Linearized_power_flow} to be
inverted, so that we have $\mathbf{v}'$ in terms of $\mathbf{s}'$, and we have
the form 
\begin{equation}\label{eq:LPFE_voltage}
    \mathbf{v}' = A\mathbf{s}' + b
\end{equation}
Comparing \eqref{eq:Linearized_power_flow} to \eqref{eq:LPFE_voltage}, we see 
that $A = \mathbf{J}^{-1}$ and 
$b = -\mathbf{J}^{-1}\mathbf{s}'_* + \mathbf{v}'_*$.

Finally, the MPC algorithm will be deciding on all elements in $\mathbf{s}'$ and
$\mathbf{v}'$ \emph{except} the complex slack voltage $v_0$, since this 
represents the voltage at the grid connection. So, we should be 
including two extra constraints: 
$\mathbf{v}'_1 = v_{0,\mathrm{grid}}\re$ and 
$\mathbf{v}'_{N+2} = v_{0,\mathrm{grid}}\im$ at the slack bus, where 
$v_{0,\mathrm{grid}}\re$ and $v_{0,\mathrm{grid}}\im$ are problem data. 
Usually, $v_{0,\mathrm{grid}}\im$ is defined as being zero (and all other
voltage angles are defined based on this).

\section*{Appendix: Power flow equation Jacobian}

Here, for clarity, we use notation
$y\re_{km} \coloneqq [\mathbf{Y}\re]_{k,m} = 
\Re([\mathbf{Y}]_{k,m})$ and likewise
for $y\im_{km}$ (the conductance and susceptance between the $k$th and $m$th
nodes). Further, let $\mathbf{y}_{k}\tran \in \mathbb{C}^{1\times (N+1)}$ be the
$k$th row of $\mathbf{Y}$. 
\emph{Note: this notation relies implicitly on $\mathbf{Y}$ 
being symmetric. If this is not the case, $\mathbf{y}_{k}\tran$ should be 
replaced with  $\mathbf{y}_{k:}$ for clarity}. 

The Jacobian matrix of $f(\cdot)$ is split into four types of terms:
$\frac{\partial {s}\re_k}{\partial {v}\re_m} =
\frac{\partial {P}_k}{\partial {v}\re_m}$, 
$\frac{\partial {s}\re_k}{\partial {v}\im_m} =
\frac{\partial {P}_k}{\partial {v}\im_m}$, 
$\frac{\partial {s}\im_k}{\partial {v}\re_m} =
\frac{\partial {Q}_k}{\partial {v}\re_m}$, and
$\frac{\partial {s}\im_k}{\partial {v}\im_m} =
\frac{\partial {Q}_k}{\partial {v}\im_m}$. 
These are found to be

\begin{subequations}\label{eq:Jacobian_derivatives}
    \begin{align}
        \frac{\partial {s}\re_k}{\partial {v}\re_m} &= 
            \left(v_k\re y_{km}\re + v_k\im y_{km}\im\right)
            + \delta_{km}\left((\mathbf{y}_{k}\re)\tran\mathbf{v}\re - 
            (\mathbf{y}_{k}\im)\tran\mathbf{v}\im \right) 
             \\
        \frac{\partial \tilde{s}\re_k}{\partial {v}\im_m} &= 
            \left(v_k\im y_{km}\re - v_k\re y_{km}\im \right)
            + \delta_{km}\left((\mathbf{y}_{k}\im)\tran \mathbf{v}\re
                + (\mathbf{y}_{k}\re)\tran \mathbf{v}\im\right)
            \label{eqref:delPdelvim}\\
        % This is the one that disagrees with Sereeter et al
        \frac{\partial \tilde{s}\im_k}{\partial {v}\re_m} &= 
            \left(v_k\im y_{km}\re - v_k\re y_{km}\im\right)
            -\delta_{km}\left((\mathbf{y}_{LL,k}\im)\tran \mathbf{v}\re
                + (\mathbf{y}_{LL,k}\re)\tran \mathbf{v}\im \right)
            \label{eqref:delQdelvre} \\ 
        \frac{\partial \tilde{s}\im_k}{\partial {v}\im_m} &= 
            -\left(v_k\re y_{km}\re + v_k\im y_{km}\im\right)
            +\delta_{km}\left((\mathbf{y}_{LL,k}\re)\tran \mathbf{v}\re
                - (\mathbf{y}_{LL,k}\im)\tran \mathbf{v}\im\right)
    \end{align}
\end{subequations}
where 
$$ \delta_{km} = \begin{cases} 1, & k=m \\ 0, & \text{otherwise} \end{cases} $$
is the Kronecker delta. I'll point out here that my derivative 
\eqref{eqref:delQdelvre} disagrees with Sereeter et al [2], Table 5, section 
`Cartesian, $i=k$', term 2. I think that 
this is a typo on their part. Also, the coded version of this in 
\texttt{power{\textunderscore}flow{\textunderscore}tools.%
full{\textunderscore}power{\textunderscore}flow{\textunderscore}Jacobian}, 
because the function uses the definition for $\mathbf{Y}'$ (which has negative
entries).

\section*{References}
\begin{itemize}
    \item [[2]] Sereeter et al. 
    https://www.sciencedirect.com/science/article/pii/S0377042719301876
\end{itemize}

\end{document}